\documentclass[ngerman]{article}

\usepackage{babel}
\usepackage[margin=2.5cm]{geometry}
\usepackage{amssymb}
\usepackage{caption} \captionsetup[table]{skip=10pt}
\usepackage[table]{xcolor}
\usepackage{tabularx}
\usepackage{float}
\usepackage{multirow}
\usepackage{makecell}
\renewcommand{\arraystretch}{1.2}
\newcolumntype{L}{X}
\newcolumntype{C}{>{\centering\arraybackslash}X}
\newcolumntype{R}{>{\raggedleft\arraybackslash}X}

\newcommand\mik{\cellcolor{green!9} Mik Müller}
\newcommand\silas{\cellcolor{blue!7} Silas Groh}
\newcommand{\chap}[1]{Kapitel \texttt{#1}}
\newcommand{\sect}[1]{Abschnitt \texttt{#1}}
\newcommand{\subsec}[1]{Unterabschnitt \texttt{#1}}

\begin{document}
    \section{Authoren}
    \begin{table}[H]
        \caption{Authoren der einzelnen Abschnitte}
        \begin{tabularx}{\linewidth}{|l|L|}
            \hline
            \rowcolor{gray!20} Abschnitt    &  Author \\ \hline
            \chap{1} & \mik \\ \hline
            \sect{2.1} & \silas \\ \hline
            \sect{2.2} & \mik \\ \hline
            \sect{3.1} & \silas \\ \hline
            \sect{3.2} & \mik \\ \hline
            \sect{4.1} & \mik \\ \hline
            \sect{4.2} & \silas \\ \hline
            \sect{4.3} & \mik \\ \hline
            \subsec{5.1.1} & \mik \\ \hline
            \subsec{5.1.2} & \mik \\ \hline
            \subsec{5.1.3} & \mik \\ \hline
            \subsec{5.1.4} & \silas \\ \hline
            \subsec{5.1.5} & \mik \\ \hline
            \subsec{5.1.6} & \mik \\ \hline
            \sect{5.2} & \mik \\ \hline
            \sect{5.3} & \silas \\ \hline
        \end{tabularx}
    \end{table}

    \begin{table}[H]
        \caption{Authoren der gezeigten Komponenten von rush}
        \begin{tabularx}{\linewidth}{|l|L|}
            \hline
            \rowcolor{gray!20} Komponente    &  Hauptauthor \\ \hline
            Lexer & \mik \\ \hline
            Parser & \silas \\ \hline
            Semantic Analyzer & \mik \\ \hline
            tree-walking interpreter & \silas \\ \hline
            VM interpreter & \mik \\ \hline
            WASM compiler & \silas \\ \hline
            LLVM compiler & \mik \\ \hline
            RISC-V compiler & \mik \\ \hline
            X86\_64compiler & \silas \\ \hline
        \end{tabularx}

        \vspace{.3cm}
        
        \fbox{\begin{tabular}{ll}
        \textcolor{green!65}{$\blacksquare$} & Mik Müller \\
        $\textcolor{blue!50}\blacksquare$ & Silas Groh 

        \end{tabular}}
    \end{table}

    \newpage

    \section{Selbstständigkeitserklärung}
    Hiermit versicheren wir, dass wir diese Arbeit selbstständig angefertigt, keine anderen als die von uns angegebenen Quellen und Hilfsmittel benutzt und die Stellen der Arbeit, die im Wortlaut oder dem Inhalt nach aus anderen Werken entnommen wurden, in jedem einzelnen Fall mit genauer Quellenangabe kenntlich gemacht habe. Verwendete Informationen aus dem Internet sind der Arbeit als Ausdruck im Anhang beigefügt. Wir sind damit einverstanden, dass die von uns verfasste Arbeit der schulinternen Öffentlichkeit in der Bibliothek der Schule zugänglich gemacht wird.

    \vspace{2cm}
    \noindent Wuppertal, 28. Februar 2022
    \vspace{10pt}
    \hrule width \textwidth \relax
    \vspace{.35cm}
    \noindent Ort, Datum \hspace{4cm} Unterschrift {\small Mik Müller} \hspace{2cm} Unterschrift {\small Silas Groh}
\end{document}
