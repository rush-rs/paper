\subsection{The Compiler Targeting the rush VM}

Since the rush VM interprets instructions directly, there must be a compiler translating the AST into these instructions.
For this purpose, we have implemented a compiler translating rush into instructions which can be understood by this VM\@.
Compared to the rush compiler targeting \emph{RISC-V},
implementation of this compiler has proven to be significantly simpler since the rush VM uses a stack-based design.
Since the VM's architecture was developed with the features of rush in mind,
the compiler sometimes requires surprisingly little effort for translating some AST-nodes.
For instance, the compiler translates infix-expressions, such as $n + m$, into instructions using the \texttt{infix\_expr} method.
A part of this method is displayed in Listing~\ref{lst:vm_compile_infix_expr}.

\TSListing[ranges={538-540}, caption={Compilation of Infix-Expressions Targeting the VM}, label={lst:vm_compile_infix_expr}, float=H]{deps/rush/crates/rush-interpreter-vm/src/compiler.rs}

Here, the left hand side expression is compiled first.
Next, the right-hand side is compiled too.
Finally, the appropriate arithmetic instruction is inserted.
The final instruction is generated by a helper function which converts an infix-operator into a matching instruction.
Most of the other compilers we have implemented for the rush project required significantly more code in order to implement the translation of infix-expressions.

\TSListing[ranges={445-450}, caption={Compilation of Expressions Targeting the VM}, label={lst:vm_compile_expr}, float=H]{deps/rush/crates/rush-interpreter-vm/src/compiler.rs}

The code in Listing~\ref{lst:vm_compile_expr} shows the top of the \texttt{expression} method in the VM compiler.
When we examine the method's signature, it becomes apparent that it only consumes an \texttt{AnalyzedExpresssion}.
However, the method does not return anything which represents the runtime value of the expression.
This is possible because the types of expression displayed in the snipped are pushed onto the stack directly.
By pushing the values of atomic expressions onto the stack directly, most tree-traversing methods do not need to return values.
Due to this, short and elegant code like the one in Listing~\ref{lst:vm_compile_infix_expr} can be implemented.
In other compilers, the method responsible for compiling expressions usually returns the register which contains the value of the compiled expression at runtime.
This way, other parts of the compiled program can still use the runtime values of compiled expressions.

The rush VM includes a special instruction for the mathematical power operation (\texttt{**}).
Since many real architectures lack such a power instruction,
implementing a rush compiler targeting the VM has proven to be less demanding in this way.
On the opposite, many other rush compilers demanded implementation of special edge-cases in order to make compiling power-expressions feasible.
Furthermore, the VM also includes an \texttt{exit} instruction which terminates the fetch-decode-execute cycle instantly.
Here, the VM would come to a halt instantly, returning the top value on the stack as its exit code.
These examples showed how a carefully chosen target architecture simplifies the implementation of its compiler by a great deal.

However, there is also one aspect of the VM which made implementation of the compiler targeting the VM more demanding than usual.
For instance, in most Assembly dialects, \emph{labels} can be used to allow jumps between blocks of code.
However, the VM intentionally does not support the use of such labels.
Since the VM would have to look up the exact instruction index of a label at runtime,
each jump targeting a label would involve some additional overhead.
This overhead is eliminated by the assembler during assembly of a program.
Since the assembler performs these lookups during translation,
the CPU does not have to deal with label lookups at runtime.
Like seen in the previous examples, jumping VM instruction require the exact index of the target instruction as their operands.
Therefore, the exact target index to which the instruction should jump must be known.
To illustrate this issue, we will consider how loops are implemented in the VM\@.
The rush code in Figure~\ref{fig:vm_loops} presents a program containing a loop.
In the loop's body, the variable \texttt{n} is incremented by 1.
Next, the \texttt{break} keyword is used to terminate loop execution.
Therefore, the total amount of iterations is 1.

\noindent
\begin{figure}[h]
	\begin{minipage}{.5\textwidth}
		\centering
        \TSListing[fancyvrb={frame=none}]{listings/vm_basic_loop.rush}
	\end{minipage}%
	\begin{minipage}{.5\textwidth}
		\centering
        \TSListing[raw=true, fancyvrb={frame=none}]{listings/vm_basic_loop_instructions.txt}
	\end{minipage}
	\caption{How Loops are Interpreted by the VM}\label{fig:vm_loops}
\end{figure}

The rush VM instructions of the \texttt{main} function are displayed on the right side of Figure~\ref{fig:vm_loops}.
Here, lines 2 and 3 are responsible for declaring the variable \texttt{n}.
The instructions in the lines 4--9 are used to increment the variable \texttt{n} by 1.
A new instruction which we have not covered so far is the \texttt{clone} instruction.
This instruction \emph{clones} the top item on the stack it without prior calls to \texttt{pop}.
Therefore, after the instruction has been executed, two identical values exist on the top of the stack.
This instruction is only used in assign-expressions in order to duplicate the address value of the assignee variable.

After \texttt{n} is incremented, the instruction in line 10 jumps to the instruction index 11.
However, the last valid index is 10, it is represented by the \texttt{jmp 3} instruction.
If this occurs, the VM has no next instruction to fetch and therefore stops its fetch-decode-execute cycle.
Since this instruction jumps to a position outside the loop, it represents the \texttt{break} statement in line 5 of the source program.
The \texttt{jmp} instruction in line 11 is responsible for the repetition introduced by the loop.
This instruction jumps to the first instruction of the loop's body in line 4.
Therefore, the instructions inside the loop's body are executed repeatedly.
The difficulty presented by this design is that the index of the jump's target instruction must be known before the target instruction is inserted.
The code in Listing~\ref{lst:rush_vm_compiler_loop} displays a part of the method responsible for compiling loops for the rush VM.

\TSListing[ranges={337-343}, caption={Implementation of Loops in the rush VM Compiler}, label={lst:rush_vm_compiler_loop}, float=H]{deps/rush/crates/rush-interpreter-vm/src/compiler.rs}

The statement in line 337 inserts the instruction responsible for jumping back to the start of the loop's body.
In line 340, the top loop is popped from the \texttt{loops} stack.
This stack is an internal field used by the compiler in order to save information about loops.
The top item on this stack always represents the loop currently traversed by the compiler.
Each loop saves two lists, each containing the indices of jump-instructions whose target index needs to be adjusted.
The first list contains the indices of jump-instructions generated by \texttt{break} statements
while the second lists saves instructions generated by \texttt{continue} statements.
For instance, if the compiler encounters a \texttt{break} statement, the code in Listing~\ref{lst:rush_vm_compiler_break} is executed.

\TSListing[ranges={268-273}, caption={Compilation of \texttt{break} Statements in the rush VM Compiler}, label={lst:rush_vm_compiler_break}, float=H]{deps/rush/crates/rush-interpreter-vm/src/compiler.rs}

Here, the \texttt{pos} variable saves the index of the jump-instruction to be inserted.
In line 271, this index is then inserted into the list containing the placeholder indices of the current loop.
Lastly, the \texttt{jmp} instruction is inserted containing a placeholder target index.
Therefore, at the end of each loop's compilation, there will be a list containing the indices of instructions whose target indices need to be adjusted.
In line 342 of Listing~\ref{lst:rush_vm_compiler_loop}, the \texttt{self.fill\_blank\_jmps} method is used to set the target indices of the specified jump-instructions to \texttt{pos}.
We will omit the explanation of this method because it only iterates over the passed list of indices, replacing the target of the jump-instruction at the current index during the process.

As a conclusion, designing and implementing a compiler targeting the rush VM has presented itself as a reasonable task.
Previously, we have presented positive and negative aspects of writing this compiler.
For instance, altering the target architecture to mitigate difficulties which occur during the implementation of the compiler has proven itself often to be extremely helpful.
