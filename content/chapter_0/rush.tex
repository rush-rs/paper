\section{Characteristics Of The Rush Programming Language}

Rush is a simple programming language which was purposely designed for this paper.
The language features a \emph{static type system}, \emph{arithmetic operators}, \emph{local and global variables}, \emph{if-else control flow}, \emph{loops}, and \emph{functions}.
\TSListing[caption={Generating Fibonacci Numbers Using rush}, label={lst:rush_fib}, float=H]{fib.rush}

In the git commit \texttt{\input "|git rev-parse --short @:./listings/rush"}, the entire rush project included
\input "|tokei ./listings/rush -o json | jq '.Total.code'" lines of source code \footnote{Lines of source code with omitted blank lines and comments}.
On the first sight, this might seem like a large number for a simple programming language.
However, the rush project includes a lexer, a parser, a semantic analyzer, five compilers, one interpreter, and several other tools, like a language server for IDE support.
Furthermore, we can compare the size of the Rust compiler with the one of the rush project.
