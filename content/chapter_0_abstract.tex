\chapter{Abstract}

Programming languages are undoubtedly of great importance for various aspects of
modern-day life. Even if they remain unnoticed, digital systems running programs
written in some sort of programming language are ubiquitous.
\newline
However, there are numerous ways of implementing a programming language. A
language designer could choose an interpreted or a compiled approach for their
language's implementation. Both ways of program execution come with their own
advantages and disadvantages.
\newline
This paper aims to inform the reader about different means of program execution.
For this, we will implement our own simple programming language called \emph{rush}.
However, we will not settle on one method of program execution. Therefore, in
order to highlight the differences between the different methods of program
execution, we will implement rush using different backends.
\newline
First, two interpreted backends are presented. One of them uses a tree-walking
approach while the other one uses a virtual machine for executing rush code.
Furthermore, two backends which compile to high-level targets are presented. As
examples for high-level targets, we chose a compiler targeting \emph{LLVM} and a
compiler targeting \emph{WebAssembly}. Furthermore, we will present two compilers
which generate low-level assembly code. For this, a compiler targeting \emph{RISC-V}
assembly and another compiler targeting \emph{X86\_64} assembly are presented.
\newline
This paper assumes that the reader has basic knowledge about computer
programming and computer hardware. Most code samples will be Rust code as the
entire project is written in Rust.
