\section*{Introduction}
\textcolor{red}{Should this heading exist?}

Programs are often formulated in complex structures which represent the logic of the underlying algorithm.
However, a computer can only interpret a sequence of CPU instructions.
Therefore, the source program has to be translated into such a sequence of target-dependent instructions before it can be executed.
The translation process is called \emph{compilation}, and is performed by program which is called \emph{a compiler}.
However, the output instruction sequence must represent the identical algorithm specified in the source code.
It is apparent that compilation requires significant effort and must obey complex rules,
since it should translate the source program precisely.

The first compiler was implemented around 1956 and aimed to translate \emph{Fortran} to computer instructions.
However, the success of this programming endeavor was not assured until the program was completed.
In total, the program involved roughly 18 man years of work
and is thereby regarded as one of the largest programming projects of the time.

