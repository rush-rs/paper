\section{RISC-V: A Compiler for a Modern RISC Architecture}
\subsection{The RISC-V Architecture}

\begin{itemize}
	\item Introduced in 2011
	\item UC Berkely research project
	\item Since then: rising in popularity
	\item ISA = \enquote{instruction set architecture}
	\item Was developed for:
	      \begin{itemize}
		      \item All sizes of processors: from embedded to high-performance computer
		      \item compatibility to popular software stacks and programming languages
		      \item serves as extendable base for customized accelerators
		      \item Should be stable
	      \end{itemize}
	\item One of the few ISAs which were developed this decade instead of the 1970s / 1980s
	\item Open ISA: unlike most previous ISAs, it is free from a bond to a single comporation
    \item Base ISA: \qVerb{RV321}.
        Frozen, will never change, gives assembly programmers and compiler writers a stable target.
    \item The base instruction set is extendable using extensions like (multiply: \qVerb{RV32M}) or (double-precision floats: \qVerb{RV32D}).
\end{itemize} \cite[Chapter~1]{Patterson2017}.

\TODO{Also interesting: \cite[p.~10]{Patterson2017}, \cite[Chapter~2]{Patterson2017}.}

\begin{itemize}
	\item Present Some example instructions
	\item Register layout \& count
	\item Calling convention
\end{itemize}
