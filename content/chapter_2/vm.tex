\section{Using A Virtual Machine}

Just like a tree-walking interpreter, a virtual machine presents a method of implementing an interpreter for a programming language.
However, the way a virtual machine operates fundamentally differs from the one which the tree-walking interpreter uses.
For rush, we have implemented a virtual machine backend in order to compare it to the previously displayed tree-walking interpreter.

\subsection{Defining A Virtual Machine}

Often, one might encounter the term \emph{virtual machine} when talking about emulating an existing type of computer using a software system.
This emulation often includes simulating devices like displays, disk or GPU.
In this context however, a \emph{virtual machine} is a software entity which emulates how a computer interprets instructions.
Just like a real computer, a virtual machine executes instructions directly.
Often, a virtual machine may make use of an architecture similar to the \emph{Von Neumann architecture}.

The von Neumann architecture was first introduced by John Neumann in the year 1945
The processing unit contains components like the \emph{ALU}\footnote{Short for \enquote{arithmetic logic unit}}, registers, and a control unit.
\cite[p.~172]{Ledin2020-yp}.
The control unit is of particular importance since it manages the instruction counter and plays a vital role in the \emph{fetch-decode-execute} cycle.

\begin{itemize}
	\item \textbf(Fetch): The processor's control unit loads the next instruction from the adequate memory location.
	      The value of the next instruction is then placed into the processor's internal instruction register.
	\item \textbf(Decode):
	      The processor's control unit examines the fetched instruction in order to determine if additional steps must be taken during instruction execution.
	      Such steps may involve the ALU performing read-write operations on registers or memory locations.
	\item \textbf(Execute):
	      The control unit invokes all necessary steps for executing the requested operation.
	      For instance, the control unit may invoke the ALU in order to perform an addition.
\end{itemize}

A computer's processor performs this fetch-decode-execute cycle repeatedly from the moment it is powered on until the point in time where it is powered down again.
For relatively simple processors, each cycle is executed in an isolated manner because instructions are executed in a sequential order.
This means that the execution of the instruction $i$ is delayed until execution of $i - 1$ has completed.
For most virtual machines, executing the input instructions in sequential order is often the simplest solution.
The \emph{IPC}\footnote{Short for \enquote{instructions per clock}} provides a performance metric for processors.
It determines how long a processor takes to execute instructions relative to its clock speed.
For instance, if the IPC is 0.25, a processor would require four clocks for one instruction to be executed.

\cite[pp.~208-209]{Ledin2020-yp}.


Unlike the tree-walking interpreter, a virtual machine does not traverse the AST.

\subsection{The rush Virtual Machine}
\subsection{How The Virtual Machine Executes A Rush Program}
\subsection{The Compiler Targeting The Virtual Machine}
