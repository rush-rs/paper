% chktex-file -2
\newpage %TODO: remove this force newpage?
\section{The rush Programming Language}

For this paper, we have developed and implemented a simple programming language called \emph{rush}\footnote{Capitalization of the name is omitted intentionally.}.
The language features a \emph{static type system}, six different data types\footnote{Including the \qVerb{int}, \qVerb{float}, \qVerb{bool}, \qVerb{char}, \emph{unit}, and \emph{never} type.}, \emph{arithmetic operators}, \emph{logical operators}, \emph{local and global variables}, \emph{pointers}, \emph{if-else expressions}, \emph{loops}, and \emph{functions}.
In order to introduce the language, we will now consider the code in Listing~\ref{lst:rush_fib}.

\Lirsting[caption={Generating Fibonacci numbers using rush.}, label={lst:rush_fib}, float=H]{listings/fib.rush}

This rush program can be used to generate numbers included in the Fibonacci sequence.
In the code, a function named \qVerb{fib} is defined using the \qVerb{fn} keyword.
This function accepts the parameter \qVerb{n}, which denotes the position in the Fibonacci sequence of the number to be calculated.
Since \qVerb{int} is specified as the type of the parameter, the function may be called using any integer value as its argument.
However, the constraint $n \in \mathbb{N}_0$ must be valid in order for this function to return the correct result\footnote{Assuming the function should comply with the original Fibonacci definition.}.
In this example, the \qVerb{main} function calls the \qVerb{fib} function using the natural number `10'.
In rush, every valid program must contain exactly one \qVerb{main} function since that is where program execution will start.
Even though rush provides a \qVerb{return} statement, the body of the \qVerb{fib} function does not contain one.
This is because blocks, like the one of the \qVerb{fib} function, return the result of their last expression.
The if-else construct is also an expression since it represents the last entity in the block, and is not followed by a semicolon.
If \qVerb{n} is less than `2', it is returned without modification.
Otherwise, the function calls itself recursively in order to calculate the sum of the preceding Fibonacci numbers `$n - 2$' and `$n - 1$'.
The result value of the entire if-expression is calculated by using one of the two branches.
Since the if-else construct is also an expression, there is no need for a redundant \qVerb{return} statement.
In line~2, the \qVerb{exit} function is called.
Even though this function is not defined anywhere, the code still executes without any errors.
This is due to the fact that the \qVerb{exit} function is a \emph{built-in} function that is used to exit a program with the specified exit code.

\subsection{Features}

As outlined previously, rush implements various features, which are also found in many other relevant programming languages today.
Generally, rush is a \emph{procedural} programming language, meaning that code is partitioned into \emph{procedures}, which are commonly referred to as \enquote{functions}.
This paradigm has been selected as it is also common across most of today's popular programming languages~\cite[p.~278]{Tucker2007-qr}.
The language provides many features which are meant to increase developer productivity.
For instance, a program might use a \qVerb{loop} to implement code repetition easily.
For reference, Table~\ref{tbl:rush_features} shows some features of the rush programming language.

Just like most other compiled languages, rush also includes a \emph{static type system},
meaning that every variable has a type, which cannot change during runtime.
For instance, after a new variable, which can hold integers, was defined,
only integer values may be assigned to it. This way, the compiler can validate the semantic
correctness of the program, showing error messages if an erroneous program is provided.
Apart from mandating a program's correctness, a static type system also makes the program easier to read for humans.
Therefore, the rush programming language is designed in a way that promotes correctness and programmer efficiency.
Table~\ref{tbl:rush_types} provides an overview of all the data types rush provides.

\begin{wraptable}{o}{.4\textwidth}
	\centering
	\caption{Lines of code of the project's components in commit \protect\rushCommit{}.}\label{tbl:rush_loc_components}
	\begin{tabularx}{\linewidth}{|L|l|}
		\hline
		\rowcolor{gray!25} Component & Lines                                            \\ \hline
		lexer and parser             & \tokei{./deps/rush/crates/rush-parser}           \\ \hline
		tree-walking interpreter     & \tokei{./deps/rush/crates/rush-interpreter-tree} \\ \hline
		VM compiler and runtime      & \tokei{./deps/rush/crates/rush-interpreter-vm}   \\ \hline
		Wasm compiler                & \tokei{./deps/rush/crates/rush-compiler-wasm}    \\ \hline
		LLVM compiler                & \tokei{./deps/rush/crates/rush-compiler-llvm}    \\ \hline
		C transpiler                 & \tokei{./deps/rush/crates/rush-transpiler-c}     \\ \hline
		\riscv{} compiler            & \tokei{./deps/rush/crates/rush-compiler-risc-v}  \\ \hline
		x64 compiler                 & \tokei{./deps/rush/crates/rush-compiler-x86-64}  \\ \hline
	\end{tabularx}
\end{wraptable}

While the first four types are self-explanatory, the last two demand special explanation.
The \emph{unit} type is used in order to denote that a function returns no value.
This concept is comparable to \emph{void} in other languages like C or Java.
In rush however, the unit type is treated like a value at runtime.
Even though the unit type can be saved inside a variable or used as a function parameter, all rush compilers simply ignore it since it holds no value.
In Listing~\ref{lst:rush_fib}, the \qVerb{main} function also returns the unit type.
A function named \qVerb{foo} can be defined like `\LirstInline{rush}{fn foo() -> () {}}', as well as `\LirstInline{rush}{fn foo() {}}', because the unit type is the default.
The \emph{never} type is used in order to denote that an expression, or statement, interrupts the normal control flow, so that the expression never yields a value.
Because the built-in \qVerb{exit} function terminates a program, its result will never be of relevance at runtime.
However, the fact that the never type is returned is relevant as this information is valuable for the semantic analysis and the compilers.
Furthermore, \qVerb{break}, \qVerb{continue}, and \qVerb{return} also result in the never type.
It is to be noted that the type cannot be used explicitly.
That means the usage of syntax like `\LirstInline{rush}{let a: ! = exit(0);}' is illegal.

\begin{table}[p]
	\caption{Most important features of the rush programming language.}\label{tbl:rush_features}
	\rowcolors{2}{gray!15}{}
	\begin{tabularx}{0.95\textwidth}{lLl}
		\rowcolor{gray!25} Name & Description                                                                 & Example                                          \\
		\hline
		unconditional loop      & Executes code repeatedly.                                                   & \LirstInline{rush}{loop { }}                     \\
		while-loop              & Like \qVerb{loop}, but checks a condition before each iteration.            & \LirstInline{rush}{while x < 5 { }}              \\
		for-loop                & Like \qVerb{while}, but executes an update-expression after each iteration. & \LirstInline{rush}{for i = 0; i < 5; i += 1 { }} \\
		if-expression           & Executes different code based on a condition.                               & \LirstInline{rush}{if true { } else { }}         \\
		function definition     & Defines a function, which can be called later.                              & \LirstInline{rush}{fn foo(n: int) { }}           \\
		infix-expression        & Performs mathematical and logical operations using two values.              & \LirstInline{rush}{1 + n; 5 ** 2}                \\
		prefix-expression       & Performs mathematical and logical operations using one value.               & \LirstInline{rush}{!false; -n}                   \\
		let-statement           & Defines a variable with an initial value.                                   & \LirstInline{rush}{let mut answer = 42;}         \\
		cast-expression         & Converts a value into a different type.                                     & \LirstInline{rush}{42 as float}                  \\
	\end{tabularx}
\end{table}

\begin{table}[p]
	\caption{Data types in the rush programming language.}\label{tbl:rush_types}
	\rowcolors{2}{gray!15}{}
	\begin{tabularx}{0.95\textwidth}{llL}
		\rowcolor{gray!25} Notation & Example                                  & Description                                                                                                                                                                                                                                       \\
		\hline
		\qVerb{int}                 & \LirstInline{rush}{let a: int = 0;}      & Values of the \emph{int} type can represent 64-bit signed integers, that is, $\left[-2^{63};\ 2^{63}-1\right]$.                                                                                                                                   \\
		\qVerb{float}               & \LirstInline{rush}{let b: float = 3.14;} & Values of the \emph{float} type can represent 64-bit double precision floating-point numbers, as defined by the \emph{IEEE 754--2008} standard.                                                                                                   \\
		\qVerb{bool}                & \LirstInline{rush}{let c: bool = true;}  & Values of the \emph{bool} type can represent either `true' or `false'.                                                                                                                                                                            \\
		\qVerb{char}                & \LirstInline{rush}{let d: char = 'a';}   & Values of the \emph{char} type can be any integer in the range $\left[0;\ 127\right]$, representing any valid ASCII character.                                                                                                                    \\
		\qVerb{()}                  & \LirstInline{rush}{let e: () = main();}  & The \emph{unit} type is an empty tuple representing nothing, comparable to \emph{void} in other languages. It is the implicit return type of functions in rush and is commonly used in functional languages like Haskell~\cite[p.~208]{Mena2019}. \\
		\qVerb{!}                   & \LirstInline{rush}{let f = exit(42);}    & The \emph{never} type can never be materialized, it exists to express that an expression diverges and therefore yields no value.                                                                                                                  \\
	\end{tabularx}
\end{table}

In the rush project, most of the previously presented stages of compilation are implemented as their own individual crates\footnote{A crate is a software library in Rust terms.}.
This way, each component of the programming language can be developed, tested, and maintained separately.
In the Git commit \rushCommit{}, the entire rush project consisted of
\tokei{./deps/rush} lines of source code\footnote{Blank lines and comments are not considered for all specifications of line count.}.
On the first sight, this might seem like a large number for a simple programming language.
However, the rush project includes a lexer, a parser, a semantic analyzer, two interpreters, four compilers, and several other tools like a language server for editor support.
Table~\ref{tbl:rush_loc_components} shows the lines of code of the project's individual components.
Here, it becomes apparent that the tree-walking interpreter contains the least amount of code, as it is the simplest.
The compilers targeting high level targets both require around 1500 lines.
As for the low-level compilers, they both require over 2000 lines of code, with x64 requiring around 500 lines more than \riscv{}.
By considering this table, the complexity of the individual components becomes apparent.
