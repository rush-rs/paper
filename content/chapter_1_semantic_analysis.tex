\chapter{Semantic Analysis}

Because the rush programming language shares its semantic rules accross all
backends, it would be cumbersome to implement semantic validation in each
backend individually. Therefore, it would be rational to implement a seperate
compilation step which is responsible for validating the source program's
semantics. Among other checks, the so-called semantic analyzer validates types
and variable references whilst performing type annotations. The last aspect is
of particular importance since all compiler backends rely on type information at
compile time. However, in order to obtain type information, the abstract syntax
tree of the source program must be traversed, performing numerous other checks
in the process. The semantic analyzer behaves exactly like this. In order to
preserve a clear boundary between the individual compilation steps, the parser
only validates the program's syntax without performing further validation.
Therefore, the analyzer traverses the abstract syntax tree previously generated
by the parser.

In order to highlight why type information is often required at compile time, we
will consider following code example. This listing displays a basic rush program
which calculates the sum of two integer values in order to use the sum as its
exit-code. In this example, the desired exit-code of the program is five.

\TSListing[caption={TODO}]{semantic_analysis_simple.rush}
% \texttt{}`rs
% fn main() {
    % let two = 2;
    % let three = 3;
    % exit(two + three)
% }
% \texttt{}`

In this example, the semantic analyzer will first check if the program contains
a \texttt{main} function. If this is not the case, the analyzer rejects the program
because it violates rush's semantic specification. Just like C or Rust, rush
requires that every program must contain exactly one \texttt{main} function. If a
programming languages requires a \texttt{main} function, top-level statements can be
completely forbidden, allowing more elegant code imports between different
source files. Furthermore, a \texttt{main} function clearly indicates the program's
start of execution which removes ambiguity. Therefore, the analyzer checks that
the \texttt{main} function takes no parameters and returns no value. However, in this
example, there is a valid \texttt{main} function. Now, the analyzer traverses the
function body of the \texttt{main} function. Since \texttt{let} statements are used to declare
new variables, the analyzer will add the variables \texttt{two} and \texttt{three} to its
current scope. However, unlike an interpreter, the analyzer does not insert the
variable's value into its scope. Instead of the concrete value, the analyzer
only considers the types of expressions. Therefore, in this example, the
analyzer remembers that the variables \texttt{two} and \texttt{three} store integer values.
This information will become much more useful when we consider line 4. Here, the
analyzer checks that the identifiers \texttt{two} and \texttt{three} refer to valid variables.
Just like in most other programming languages, rush does not allow the addition
of two boolean values for example. Here, the analyzer checks that the operands
of the `+` operator have the same type and that this type is valid in additions.
Because this validation requires information about types, the analyzer accesses
its scope when looking up the identifiers \texttt{two} and \texttt{three}. Since those names
were previously mapped to the \texttt{int} type, the analyzer is now aware of the
operand types and can check their validity. In this case, calculating the
sum of two integers is legal and results in another integer value. Since rush's
semantic specification states that the \texttt{exit} function requires exactly one
integer parameter, the analyzer has to check that it is called correctly.
Furthermore, the analyzer validates all function calls and declarations, not
just the ones of builtin functions. Since the result of the addition is also an
integer, the analyzer accepts this program since both its syntax and semantics
are valid.

As indicated previously, most compilers require type information whilst
generating target code. For simplicity, we will consider a fictional compiler
which can compile both integer and float additions. However, the fictional
target machine requires different instructions for addition depending on the
type of the operands. Therefore, integer addition uses the \texttt{intadd} instruction
while float addition uses the \texttt{floatadd} instruction. Here, type ambiguity would
cause difficulties. If there was no semantic analysis step, the compiler would
have to implement its own way of determining the types of the operands at
compile time. However, determining these types requires a complete
tree-traversal of the operand expressions. Due to the recursive design of the
abstract syntax tree, implementing this tree-traversal would require a
significant amount of source code in the compiler. However, the implementation
of this algorithm would be nearly identical accross all of rush's compiler
backends. Therefore, implementing type determination in each backend
individually would enlarge the compiler source code, thus making it harder to
understand. Since code duplication is considered inelegant, outsourcing this
algorithm into a separate component is likely the best option. As a result of
this, the semantic analyzer implements such a tree-traversal algorithm for
determining the types of subtrees. Because of the previously mentioned reasons,
rush's semantic analyzer also annotates the abstract syntax tree with type
information so that it is usable for later steps of compilation.

In order to obtain a better understanding of how the analyzer works, we will now
consider parts of its implementation and how they behave when analyzing the
example from above.

%TODO: continue with this here
