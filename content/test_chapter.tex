% chktex-file -2
\chapter{AAA}
\section{BBB}
\subsection{CCC}

% \lipsum[1]

% \input{|"ts2tex code/test.hms"}

\TSListing[last line=2, caption={Test caption for non-float}]{test.rush}
\TSListing[first line=3, last line=12, caption={Test caption with $math$ part}, label={lst:test}, float=H]{test.rush}

\TSListing[first line=13, caption={test2}]{test.rush}

\TSListing[caption={Full file, not float}]{test.rush}

\TSListing[caption={Rust Code}]{ts2tex.rs}

\TSListing[caption={Rush Code}]{test.rush}

\TSListing[caption={EBNF grammar notation}]{grammar.ebnf}

\TSListing[caption={LLVM IR code}, raw queries=true]{test.ll}

\TSListing[caption={WebAssemply Text}, float=H, raw=true, last line=10]{test.wat}

% \TSListing[caption={RISC-V ASM}, float=H, raw=true, last line=20]{riscv.asm}

\AnsiListing[float=h]{analyzer.txt}

\begin{table}
    \centering
    \caption{Test table caption}
    \begin{tabular}{|l|c|r|}
        \hline
        asda & asd & lasd \\
        \hline
    \end{tabular}
\end{table}

asd~\ref{lst:test}

\lipsum[2-3]

\subsection{Table}
\begin{table}[H]
    \caption{A nice table}
    \centering
    \begin{tabularx}{0.8\textwidth}{|>{\columncolor{black!15}}L|C|R|}
        \hline
        \rowcolor{black!15} & Column 1 & Column 2 \\
        \hline
        Row 1 & Content 1 1 & Content 2 1 \\ \hline
              & \multicolumn{2}{c|}{Content 1 2 und Content 2 2} \\ \cline{2-3}
        \multirow{-2}{*}{Row 2 und Row 3} & Content 1 3 & Content 2 3 \\ \hline
    \end{tabularx}
\end{table}

