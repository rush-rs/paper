\newpage
\section{Transpilers}

One more possible target for compilers is another high-level programming language.
A compiler translating one language to another high-level language is usually called a \emph{transpiler}.
Historically, the very first version of the C++ compiler was a transpiler which translated C++ into C.
The main advantage of a transpiler is that its implementation is usually very simple, as low-level compilation steps can be omitted.
A significant downside of a transpiler is that it often generates very inefficient code.
Furthermore, the entire project would also depend on the target language~\cite[p.~5]{Jeffery2021}.

For this paper, we have implemented a transpiler which generates C code from rush source programs.
Listing~\ref{lst:rush_transpiler_block} shows a rush program which terminates with the code `42'.
For this, the value of the block-expression in lines~2--5 is saved in the variable \qVerb{num}.
In the block, a temporary variable with a value of `40' is created.
The block results in a value of `42' since `2' is added to this variable.
In line~6, the \qVerb{exit} function is called, using \qVerb{num} as its argument.
Listing~\ref{lst:c_transpiler_block} shows the identical program represented using C.
The code was generated from the rush program in Listing~\ref{lst:rush_transpiler_block} using the rush C transpiler\footnote{Generated using rush (Git commit \rushCommit{}).}.

\noindent
\begin{minipage}{.45\textwidth}
	\centering
	\Lirsting[caption={A rush program containing a block-expression.}, label={lst:rush_transpiler_block}, float=H]{listings/simple_scope.rush}
\end{minipage}%
\hfill%
\begin{minipage}{.45\textwidth}
	\centering
	\Lirsting[caption={Transpiler output of the rush program in Listing~\ref{lst:rush_transpiler_block}.}, label={lst:c_transpiler_block}, float=H]{listings/generated/rush_simple_scope.c}
	\vspace{.1cm}
\end{minipage}

When examining the C output, it becomes apparent that the structure of the code has changed slightly.
In lines~1 and 2, the file contains `\texttt{\#include}' directives in order to import certain functions and types from a C standard library.
In line~4, the \qVerb{main} function is declared, but not yet defined.
The transpiler inserts function declarations at the start in order to allow rush's arbitrary order of function definitions.
Furthermore, the C program also does not include the aforementioned block because C does not have a structure comparable to rush's block-expressions.
Lines~2--5 in the rush code are represented by lines~7--8 in the C code.
This examples demonstrates how a transpiler often alters the program's structure in order to represent the source program in another language with different semantic rules.
Since transpilers are often not the best way of implementing a compiler, the internals of this transpiler will not be discussed further.
To conclude, transpilers can present an easy but often impractical way of implementing a compiler.
Due to its disadvantages, a transpiler is often only considered during the prototype phase of compiler implementation~\cite[p.~5]{Jeffery2021}.
It is to be mentioned that some successful and popular transpilers exist.
For instance, the popular programming language \emph{Typescript} uses a transpiler to generate Javascript~\cite[Chapter~2]{Cherny2019-rd}.
