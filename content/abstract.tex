\begin{abstract}
	\section*{Abstract}

	Programming languages are undoubtedly of great importance for various aspects of
	modern-day life. Even if they remain unnoticed, digital systems running programs
	written in some sort of programming language are ubiquitous.
	However, there are numerous ways of implementing a programming language.
	A language designer could choose an interpreted or a compiled approach for their
	language's implementation. Both ways of program execution come with their own
	advantages and disadvantages.

	This paper aims to inform the reader about different means of program execution, focussing on compiler construction.
	However, we will only focus on the basics since implementing a programming language is often a demanding task.
	In order to include practical examples, we will explain concepts on the basis of \emph{rush}, our own programming language, which was purposefully designed for this paper.
	During the implementation of rush, the focus for this paper has shifted slightly.
	As the title suggests, we originally planned to only implement and explain one compiler.
	However, there are numerous architectures which a compiler could target and settling on just one felt like the reader would miss out on too much.
	Therefore, we have implemented rush using two interpreters, one transpiler, and four compilers.

    In Chapter~\ref{chap:introduction}, we will give an introduction to implementing a programming language.
	Moreover, the rush programming language and its characteristics are presented.
    In Chapter~\ref{chap:analyzing_source}, the process of analyzing the program's syntax and semantics is explained.
    Chapter~\ref{chap:interpreting} focuses on how interpreters can be used in order to implement a programming language.
	Here, we will differentiate between a \emph{tree-walking interpreter} and a \emph{virtual machine}.
    Chapter~\ref{chap:high_level_targets} illustrates how compilation to \emph{high-level} targets\footnote{High-level targets: in this case machine independent; many abstractions are provided.} works.
	As examples for high-level targets, we will present a compiler targeting the rush virtual machine, a compiler targeting \emph{WebAssembly},
	and a compiler which uses the \emph{LLVM} framework.
    Chapter~\ref{chap:low_level_targets} focuses on how compilers targeting \emph{low-level} targets\footnote{Low-level targets: specific to one CPU architecture and operating system; little abstraction is provided.} can be implemented.
	For this, we will present a compiler targeting \emph{\riscv{}} assembly and another compiler targeting \emph{x86\_64} assembly.
    Lastly, Chapter~\ref{chap:final_thoughts} presents final thoughts and conclusions.
\end{abstract}
