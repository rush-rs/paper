\section{RISC-V: A Compiler for a Modern RISC Architecture}
\subsection{The RISC-V Architecture}

\begin{itemize}
	\item Introduced in 2011 X
	\item UC Berkely research project X
	\item Since then: rising in popularity X
	\item ISA = \enquote{instruction set architecture} X
	\item Was developed for: NO
	      \begin{itemize}
		      \item All sizes of processors: from embedded to high-performance computer
		      \item compatibility to popular software stacks and programming languages
		      \item serves as extendable base for customized accelerators
		      \item Should be stable
	      \end{itemize}
	\item One of the few ISAs which were developed this decade instead of the 1970s / 1980s X
	\item Open ISA: unlike most previous ISAs, it is free from a bond to a single comporation X
    \item Base ISA: \qVerb{RV321}.
        Frozen, will never change, gives assembly programmers and compiler writers a stable target.
    \item The base instruction set is extendable using extensions like (multiply: \qVerb{RV32M}) or (double-precision floats: \qVerb{RV32D}).
    \item \enquote{RISC-V is a recent, clean-slate, minimalist, and open ISA informed by mistakes of past ISAs. The goal of the RISC-V architects is for it to be effective for all computing devices, from the smallest to the fastest. Following von Neumann’s 70-year-old advice, this ISA emphasizes simplicity to keep costs low while having plenty of registers and transparent instruction speed to help compilers and assembly language programmers map actually important problems to appropriate, quick code.}\cite[p.~11]{Patterson2017}
\end{itemize} \cite[Chapter~1]{Patterson2017}.

\TODO{Also interesting: \cite[p.~10]{Patterson2017}, \cite[Chapter~2]{Patterson2017}.}

\begin{itemize}
    \item Present Some example instructions (Refer to chapter 3 of the RISC-V READER)
	\item Register layout \& count
        RISC-V has 32 registers, ARM-32 16, x86\_32 has 8\\
        Assembly includes all extensions\\
        Concepts of \emph{pseudoinstructions}~\cite[p.~10]{Patterson2017}\\
        Has basic instructions for add, sub, and logical operations\\
        \emph{Check for all relationships between two registers, some conditional expressions involve rela- tionships between many pairs of registers. The compiler or assembly language programmer could use slt and the logical instructions and, or, xor to resolve more elaborate conditional expressions. The two remaining integer computation instructions Figure 2.1 help with assembly and}
        r0 → constant 0 register\\
        ~\cite[p.~18]{Patterson2017}
    \item ASM directives: \cite[p.~39]{Patterson2017}
    \item ASM includes 60 pseudoinstructions: \cite[p.~42]{Patterson2017}
    \item Another 32 Float registers, float load / store instructions: \cite[pp.~48f.]{Patterson2017}
    \item Refer to stack figure: \cite[p.~40]{Patterson2017}
	\item Calling convention
\end{itemize}

\newpage

The \emph{RISC-V} \emph{ISA}\footnote{Short for: \enquote{instruction set architecture}} is a new and modern \emph{reduced instruction set} architecture focussed on simplicity and extendability.
It was originally developed at \emph{UC Berkely} in the context of a research project.
Since its initial introduction in 2011, the architecture has been rapidly growing in popularity.
During that time, it was managed and led by the \emph{RISC-V foundation}, consisting of many individuals contributing to the project.
Today, the corporate members of the RISC-V foundation include companies like \emph{Google}, \emph{Microsoft}, \emph{Samsung}, and \emph{IBM}.
Therefore, the popularity and commercial attractively of the technology is apparent.
However, unlike most previous ISAs, the RISC-V architecture is a completely open-source project and therefore not controlled by a single large corporate entity.
Unlike most of the previous ISAs which were developed during the 1970s or 80s, RISC-V is one of the few ISAs which were developed this decade.

\noindent
\begin{figure}[h]
	\begin{minipage}{.4\textwidth}
		\Lirsting[fancyvrb={frame=none}]{listings/riscv_simple.rush}
	\end{minipage}%
	\begin{minipage}{.6\textwidth}
		\Lirsting[raw=true, fancyvrb={frame=none}]{listings/generated/riscv_simple.s}
	\end{minipage}
	\caption{Rush Program Alongside its RISC-V Output}\label{fig:rush_riscv_split}
\end{figure}

The rush program on the left side of Figure~\ref{fig:rush_riscv_split} shows a program which exists.
In line 2, a mutable variable named \qVerb{m} is defined using the initial value 42.
Next, the \qVerb{exit} function is called using \qVerb{m} as its argument.
Therefore, the program should terminate using 42 as its exit-code.

\begin{itemize}
    \item .global \_start in line 1
    \item .section .text in line 2
    \item Line 5: \_start label as program entry
    \item calling the \qVerb{main} function in line 6
    \item Line 7, placing the exit-code into register a0 (a0 → first int call argument)
    \item calling exit in line 8
\end{itemize}

\subsubsection{Register Layout}



\subsubsection{Calling Convention}

%\enquote{RISC-V is a recent, clean-slate, minimalist, and open ISA informed by mistakes of past ISAs. The goal of the RISC-V architects is for it to be effective for all computing devices, from the smallest to the fastest. Following von Neumann’s 70-year-old advice, this ISA emphasizes simplicity to keep costs low while having plenty of registers and transparent instruction speed to help compilers and assembly language programmers map actually important problems to appropriate, quick code.}\cite[p.~11]{Patterson2017}
