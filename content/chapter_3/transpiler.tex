\section{Transpilers}

Another, more esoteric target of a compiler is presented by another high-level programming language.
A compiler translating one language to another language is usually called a \emph{transpiler}.
Historically, the very first version of the C++ compiler was a transpiler which translated C++ into C.
The main advantage of a transpiler is that its implementation is usually very simple as low-level compilation steps can be omitted.
A significant downside of a transpiler is often that it generates very inefficient code.
Furthermore, the entire project would also dependent on the target programming language~\cite[p.~5]{Jeffery2021}.

For this paper, we have implemented a transpiler which generates \emph{ANSI C} from rush source programs.
Listing~\ref{lst:rush_transpiler_block} shows a rush program which terminates using 42 as its exit code.
For this, the value of the block-expression in the lines 2--5 is saved in the variable \qVerb{num}.
In the block, a temporary variable with a value of 40 is created.
The block results in a value of 42 since 2 is added to the value of the temporary variable.
In line 6, the \qVerb{exit} function is called, using \qVerb{num} as its argument.
On the right side, Listing~\ref{lst:c_transpiler_block} shows the identical program represented using ANSI C.
The code was generated from the rush program in Listing~\ref{lst:rush_transpiler_block} using the rush C compiler\footnote{Generated in Git commit \rushCommit, automatically built with this document}.

\begin{minipage}{.34\textwidth}
	\center
	\Lirsting[caption={Rush Program Containing a Block-Expression}, label={lst:rush_transpiler_block}, float=H]{listings/simple_scope.rush}
\end{minipage}%
\hspace{3cm}%
\begin{minipage}{.45\textwidth}
	\center
	\Lirsting[caption={C Output of Rush Program in Listing~\ref{lst:rush_transpiler_block}}, label={lst:c_transpiler_block}, float=H]{listings/generated/rush_simple_scope.c}
\end{minipage}

When examining the C output, it becomes apparent that the structure of the code has changed slightly.
In line 1 and 2, the file contains `\texttt{\#include}' directives in order to import files from the C standard library.
In line 4, the \qVerb{main} function is declared, but not defined.
Although this might not seem very rational in this example,
declaring functions before their declaration allows rush functions to be called before their definition.
Furthermore, the C program also does not include the previously mentioned block.
The lines 2--5 in the rush code are represented by the lines 7 and 8 in the C code because C does not have a structure comparable to rush's block-expressions.
This examples demonstrates how a transpiler often alters the program's structure in order to represent the source program in another language.
Due to the practical irrelevance of transpilers, the internals of this transpiler will not be discussed further.
As a conclusion, transpilers can present an easy but limited way of implementing a compiler.
Due to its disadvantages, a transpiler is often only considered during the prototype phase of compiler implementation.
