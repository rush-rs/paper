\section{Compilation to WebAssembly}

\begin{enumerate}
    \item what is WASM and why
    \item modules
    \begin{itemize}
        \item binary and text format (sections, leb128)
        \item globals, functions, imports, exports, \ldots
        \item uncommon to target binary format
    \end{itemize}
    \item WASI
    \item basic implementation
    \begin{itemize}
        \item leb128
        \item sections
        \item \Verb{self.function_body}
    \end{itemize}
    \item example program
\end{enumerate}

The first `external' compilation target presented here is \emph{WebAssembly}, or \emph{WASM} for short.
\TODO{research origins and goals of WASM and explain them here}
Unlike the name implies, WebAssembly is not only used in web applications.
By itself, it is only a specification that can be implemented by runtimes in any context.
Most modern browsers include such a WebAssembly runtime, but there are also standalone ones, for example \emph{wasmtime} and \emph{wasmer}.

\subsection{WebAssembly Modules}

Every valid WebAssembly file must contain exactly one module.
\TODO{confirm}
The WebAssembly specification defines two different representations for these modules.
First, there is a human-readable text representation, called \emph{WAT}\footnote{WebAssembly Text}, closely resembling S-Expressions\footnote{\TODO{what are S-Expressions}}.
This is comparable to assembly languages for CPU architectures and is the typical target for compilers.
Secondly, WebAssembly modules can also be represented using its binary format, which is optimized for size and comparable to the final binary files produced by assemblers.
Most often these binary modules are constructed from a text module by using a tool such as \emph{wat2wasm} from the \emph{WebAssembly Binary Toolkit (WABT)}.
However, the rush WebAssembly compiler instead opts to target the binary format directly, highlighting a few reasons for why most compilers should not do this.
Listing~\ref{lst:wat_demo_wat} and Listing~\ref{lst:wat_demo_hex} on page~\pageref{lst:wat_demo_wat} show the same basic WebAssembly module once as WAT and once as a commented hex dump of the same module in its binary representation as produced by \url{https://webassembly.github.io/wabt/demo/wat2wasm/}.

\Lirsting[float=p, label={lst:wat_demo_wat}, caption={Simple WebAssembly Module in Text Representation}]{listings/wat_demo.wat}
\Lirsting[float=p, label={lst:wat_demo_hex}, caption={Simple WebAssembly Module in Binary Representation}]{listings/wat_demo.hexdump}
